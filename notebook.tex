
% Default to the notebook output style

    


% Inherit from the specified cell style.




    
\documentclass[11pt]{article}

    
    
    \usepackage[T1]{fontenc}
    % Nicer default font (+ math font) than Computer Modern for most use cases
    \usepackage{mathpazo}

    % Basic figure setup, for now with no caption control since it's done
    % automatically by Pandoc (which extracts ![](path) syntax from Markdown).
    \usepackage{graphicx}
    % We will generate all images so they have a width \maxwidth. This means
    % that they will get their normal width if they fit onto the page, but
    % are scaled down if they would overflow the margins.
    \makeatletter
    \def\maxwidth{\ifdim\Gin@nat@width>\linewidth\linewidth
    \else\Gin@nat@width\fi}
    \makeatother
    \let\Oldincludegraphics\includegraphics
    % Set max figure width to be 80% of text width, for now hardcoded.
    \renewcommand{\includegraphics}[1]{\Oldincludegraphics[width=.8\maxwidth]{#1}}
    % Ensure that by default, figures have no caption (until we provide a
    % proper Figure object with a Caption API and a way to capture that
    % in the conversion process - todo).
    \usepackage{caption}
    \DeclareCaptionLabelFormat{nolabel}{}
    \captionsetup{labelformat=nolabel}

    \usepackage{adjustbox} % Used to constrain images to a maximum size 
    \usepackage{xcolor} % Allow colors to be defined
    \usepackage{enumerate} % Needed for markdown enumerations to work
    \usepackage{geometry} % Used to adjust the document margins
    \usepackage{amsmath} % Equations
    \usepackage{amssymb} % Equations
    \usepackage{textcomp} % defines textquotesingle
    % Hack from http://tex.stackexchange.com/a/47451/13684:
    \AtBeginDocument{%
        \def\PYZsq{\textquotesingle}% Upright quotes in Pygmentized code
    }
    \usepackage{upquote} % Upright quotes for verbatim code
    \usepackage{eurosym} % defines \euro
    \usepackage[mathletters]{ucs} % Extended unicode (utf-8) support
    \usepackage[utf8x]{inputenc} % Allow utf-8 characters in the tex document
    \usepackage{fancyvrb} % verbatim replacement that allows latex
    \usepackage{grffile} % extends the file name processing of package graphics 
                         % to support a larger range 
    % The hyperref package gives us a pdf with properly built
    % internal navigation ('pdf bookmarks' for the table of contents,
    % internal cross-reference links, web links for URLs, etc.)
    \usepackage{hyperref}
    \usepackage{longtable} % longtable support required by pandoc >1.10
    \usepackage{booktabs}  % table support for pandoc > 1.12.2
    \usepackage[inline]{enumitem} % IRkernel/repr support (it uses the enumerate* environment)
    \usepackage[normalem]{ulem} % ulem is needed to support strikethroughs (\sout)
                                % normalem makes italics be italics, not underlines
    

    
    
    % Colors for the hyperref package
    \definecolor{urlcolor}{rgb}{0,.145,.698}
    \definecolor{linkcolor}{rgb}{.71,0.21,0.01}
    \definecolor{citecolor}{rgb}{.12,.54,.11}

    % ANSI colors
    \definecolor{ansi-black}{HTML}{3E424D}
    \definecolor{ansi-black-intense}{HTML}{282C36}
    \definecolor{ansi-red}{HTML}{E75C58}
    \definecolor{ansi-red-intense}{HTML}{B22B31}
    \definecolor{ansi-green}{HTML}{00A250}
    \definecolor{ansi-green-intense}{HTML}{007427}
    \definecolor{ansi-yellow}{HTML}{DDB62B}
    \definecolor{ansi-yellow-intense}{HTML}{B27D12}
    \definecolor{ansi-blue}{HTML}{208FFB}
    \definecolor{ansi-blue-intense}{HTML}{0065CA}
    \definecolor{ansi-magenta}{HTML}{D160C4}
    \definecolor{ansi-magenta-intense}{HTML}{A03196}
    \definecolor{ansi-cyan}{HTML}{60C6C8}
    \definecolor{ansi-cyan-intense}{HTML}{258F8F}
    \definecolor{ansi-white}{HTML}{C5C1B4}
    \definecolor{ansi-white-intense}{HTML}{A1A6B2}

    % commands and environments needed by pandoc snippets
    % extracted from the output of `pandoc -s`
    \providecommand{\tightlist}{%
      \setlength{\itemsep}{0pt}\setlength{\parskip}{0pt}}
    \DefineVerbatimEnvironment{Highlighting}{Verbatim}{commandchars=\\\{\}}
    % Add ',fontsize=\small' for more characters per line
    \newenvironment{Shaded}{}{}
    \newcommand{\KeywordTok}[1]{\textcolor[rgb]{0.00,0.44,0.13}{\textbf{{#1}}}}
    \newcommand{\DataTypeTok}[1]{\textcolor[rgb]{0.56,0.13,0.00}{{#1}}}
    \newcommand{\DecValTok}[1]{\textcolor[rgb]{0.25,0.63,0.44}{{#1}}}
    \newcommand{\BaseNTok}[1]{\textcolor[rgb]{0.25,0.63,0.44}{{#1}}}
    \newcommand{\FloatTok}[1]{\textcolor[rgb]{0.25,0.63,0.44}{{#1}}}
    \newcommand{\CharTok}[1]{\textcolor[rgb]{0.25,0.44,0.63}{{#1}}}
    \newcommand{\StringTok}[1]{\textcolor[rgb]{0.25,0.44,0.63}{{#1}}}
    \newcommand{\CommentTok}[1]{\textcolor[rgb]{0.38,0.63,0.69}{\textit{{#1}}}}
    \newcommand{\OtherTok}[1]{\textcolor[rgb]{0.00,0.44,0.13}{{#1}}}
    \newcommand{\AlertTok}[1]{\textcolor[rgb]{1.00,0.00,0.00}{\textbf{{#1}}}}
    \newcommand{\FunctionTok}[1]{\textcolor[rgb]{0.02,0.16,0.49}{{#1}}}
    \newcommand{\RegionMarkerTok}[1]{{#1}}
    \newcommand{\ErrorTok}[1]{\textcolor[rgb]{1.00,0.00,0.00}{\textbf{{#1}}}}
    \newcommand{\NormalTok}[1]{{#1}}
    
    % Additional commands for more recent versions of Pandoc
    \newcommand{\ConstantTok}[1]{\textcolor[rgb]{0.53,0.00,0.00}{{#1}}}
    \newcommand{\SpecialCharTok}[1]{\textcolor[rgb]{0.25,0.44,0.63}{{#1}}}
    \newcommand{\VerbatimStringTok}[1]{\textcolor[rgb]{0.25,0.44,0.63}{{#1}}}
    \newcommand{\SpecialStringTok}[1]{\textcolor[rgb]{0.73,0.40,0.53}{{#1}}}
    \newcommand{\ImportTok}[1]{{#1}}
    \newcommand{\DocumentationTok}[1]{\textcolor[rgb]{0.73,0.13,0.13}{\textit{{#1}}}}
    \newcommand{\AnnotationTok}[1]{\textcolor[rgb]{0.38,0.63,0.69}{\textbf{\textit{{#1}}}}}
    \newcommand{\CommentVarTok}[1]{\textcolor[rgb]{0.38,0.63,0.69}{\textbf{\textit{{#1}}}}}
    \newcommand{\VariableTok}[1]{\textcolor[rgb]{0.10,0.09,0.49}{{#1}}}
    \newcommand{\ControlFlowTok}[1]{\textcolor[rgb]{0.00,0.44,0.13}{\textbf{{#1}}}}
    \newcommand{\OperatorTok}[1]{\textcolor[rgb]{0.40,0.40,0.40}{{#1}}}
    \newcommand{\BuiltInTok}[1]{{#1}}
    \newcommand{\ExtensionTok}[1]{{#1}}
    \newcommand{\PreprocessorTok}[1]{\textcolor[rgb]{0.74,0.48,0.00}{{#1}}}
    \newcommand{\AttributeTok}[1]{\textcolor[rgb]{0.49,0.56,0.16}{{#1}}}
    \newcommand{\InformationTok}[1]{\textcolor[rgb]{0.38,0.63,0.69}{\textbf{\textit{{#1}}}}}
    \newcommand{\WarningTok}[1]{\textcolor[rgb]{0.38,0.63,0.69}{\textbf{\textit{{#1}}}}}
    
    
    % Define a nice break command that doesn't care if a line doesn't already
    % exist.
    \def\br{\hspace*{\fill} \\* }
    % Math Jax compatability definitions
    \def\gt{>}
    \def\lt{<}
    % Document parameters
    \title{1\_network\_properties}
    
    
    

    % Pygments definitions
    
\makeatletter
\def\PY@reset{\let\PY@it=\relax \let\PY@bf=\relax%
    \let\PY@ul=\relax \let\PY@tc=\relax%
    \let\PY@bc=\relax \let\PY@ff=\relax}
\def\PY@tok#1{\csname PY@tok@#1\endcsname}
\def\PY@toks#1+{\ifx\relax#1\empty\else%
    \PY@tok{#1}\expandafter\PY@toks\fi}
\def\PY@do#1{\PY@bc{\PY@tc{\PY@ul{%
    \PY@it{\PY@bf{\PY@ff{#1}}}}}}}
\def\PY#1#2{\PY@reset\PY@toks#1+\relax+\PY@do{#2}}

\expandafter\def\csname PY@tok@gd\endcsname{\def\PY@tc##1{\textcolor[rgb]{0.63,0.00,0.00}{##1}}}
\expandafter\def\csname PY@tok@gu\endcsname{\let\PY@bf=\textbf\def\PY@tc##1{\textcolor[rgb]{0.50,0.00,0.50}{##1}}}
\expandafter\def\csname PY@tok@gt\endcsname{\def\PY@tc##1{\textcolor[rgb]{0.00,0.27,0.87}{##1}}}
\expandafter\def\csname PY@tok@gs\endcsname{\let\PY@bf=\textbf}
\expandafter\def\csname PY@tok@gr\endcsname{\def\PY@tc##1{\textcolor[rgb]{1.00,0.00,0.00}{##1}}}
\expandafter\def\csname PY@tok@cm\endcsname{\let\PY@it=\textit\def\PY@tc##1{\textcolor[rgb]{0.25,0.50,0.50}{##1}}}
\expandafter\def\csname PY@tok@vg\endcsname{\def\PY@tc##1{\textcolor[rgb]{0.10,0.09,0.49}{##1}}}
\expandafter\def\csname PY@tok@vi\endcsname{\def\PY@tc##1{\textcolor[rgb]{0.10,0.09,0.49}{##1}}}
\expandafter\def\csname PY@tok@vm\endcsname{\def\PY@tc##1{\textcolor[rgb]{0.10,0.09,0.49}{##1}}}
\expandafter\def\csname PY@tok@mh\endcsname{\def\PY@tc##1{\textcolor[rgb]{0.40,0.40,0.40}{##1}}}
\expandafter\def\csname PY@tok@cs\endcsname{\let\PY@it=\textit\def\PY@tc##1{\textcolor[rgb]{0.25,0.50,0.50}{##1}}}
\expandafter\def\csname PY@tok@ge\endcsname{\let\PY@it=\textit}
\expandafter\def\csname PY@tok@vc\endcsname{\def\PY@tc##1{\textcolor[rgb]{0.10,0.09,0.49}{##1}}}
\expandafter\def\csname PY@tok@il\endcsname{\def\PY@tc##1{\textcolor[rgb]{0.40,0.40,0.40}{##1}}}
\expandafter\def\csname PY@tok@go\endcsname{\def\PY@tc##1{\textcolor[rgb]{0.53,0.53,0.53}{##1}}}
\expandafter\def\csname PY@tok@cp\endcsname{\def\PY@tc##1{\textcolor[rgb]{0.74,0.48,0.00}{##1}}}
\expandafter\def\csname PY@tok@gi\endcsname{\def\PY@tc##1{\textcolor[rgb]{0.00,0.63,0.00}{##1}}}
\expandafter\def\csname PY@tok@gh\endcsname{\let\PY@bf=\textbf\def\PY@tc##1{\textcolor[rgb]{0.00,0.00,0.50}{##1}}}
\expandafter\def\csname PY@tok@ni\endcsname{\let\PY@bf=\textbf\def\PY@tc##1{\textcolor[rgb]{0.60,0.60,0.60}{##1}}}
\expandafter\def\csname PY@tok@nl\endcsname{\def\PY@tc##1{\textcolor[rgb]{0.63,0.63,0.00}{##1}}}
\expandafter\def\csname PY@tok@nn\endcsname{\let\PY@bf=\textbf\def\PY@tc##1{\textcolor[rgb]{0.00,0.00,1.00}{##1}}}
\expandafter\def\csname PY@tok@no\endcsname{\def\PY@tc##1{\textcolor[rgb]{0.53,0.00,0.00}{##1}}}
\expandafter\def\csname PY@tok@na\endcsname{\def\PY@tc##1{\textcolor[rgb]{0.49,0.56,0.16}{##1}}}
\expandafter\def\csname PY@tok@nb\endcsname{\def\PY@tc##1{\textcolor[rgb]{0.00,0.50,0.00}{##1}}}
\expandafter\def\csname PY@tok@nc\endcsname{\let\PY@bf=\textbf\def\PY@tc##1{\textcolor[rgb]{0.00,0.00,1.00}{##1}}}
\expandafter\def\csname PY@tok@nd\endcsname{\def\PY@tc##1{\textcolor[rgb]{0.67,0.13,1.00}{##1}}}
\expandafter\def\csname PY@tok@ne\endcsname{\let\PY@bf=\textbf\def\PY@tc##1{\textcolor[rgb]{0.82,0.25,0.23}{##1}}}
\expandafter\def\csname PY@tok@nf\endcsname{\def\PY@tc##1{\textcolor[rgb]{0.00,0.00,1.00}{##1}}}
\expandafter\def\csname PY@tok@si\endcsname{\let\PY@bf=\textbf\def\PY@tc##1{\textcolor[rgb]{0.73,0.40,0.53}{##1}}}
\expandafter\def\csname PY@tok@s2\endcsname{\def\PY@tc##1{\textcolor[rgb]{0.73,0.13,0.13}{##1}}}
\expandafter\def\csname PY@tok@nt\endcsname{\let\PY@bf=\textbf\def\PY@tc##1{\textcolor[rgb]{0.00,0.50,0.00}{##1}}}
\expandafter\def\csname PY@tok@nv\endcsname{\def\PY@tc##1{\textcolor[rgb]{0.10,0.09,0.49}{##1}}}
\expandafter\def\csname PY@tok@s1\endcsname{\def\PY@tc##1{\textcolor[rgb]{0.73,0.13,0.13}{##1}}}
\expandafter\def\csname PY@tok@dl\endcsname{\def\PY@tc##1{\textcolor[rgb]{0.73,0.13,0.13}{##1}}}
\expandafter\def\csname PY@tok@ch\endcsname{\let\PY@it=\textit\def\PY@tc##1{\textcolor[rgb]{0.25,0.50,0.50}{##1}}}
\expandafter\def\csname PY@tok@m\endcsname{\def\PY@tc##1{\textcolor[rgb]{0.40,0.40,0.40}{##1}}}
\expandafter\def\csname PY@tok@gp\endcsname{\let\PY@bf=\textbf\def\PY@tc##1{\textcolor[rgb]{0.00,0.00,0.50}{##1}}}
\expandafter\def\csname PY@tok@sh\endcsname{\def\PY@tc##1{\textcolor[rgb]{0.73,0.13,0.13}{##1}}}
\expandafter\def\csname PY@tok@ow\endcsname{\let\PY@bf=\textbf\def\PY@tc##1{\textcolor[rgb]{0.67,0.13,1.00}{##1}}}
\expandafter\def\csname PY@tok@sx\endcsname{\def\PY@tc##1{\textcolor[rgb]{0.00,0.50,0.00}{##1}}}
\expandafter\def\csname PY@tok@bp\endcsname{\def\PY@tc##1{\textcolor[rgb]{0.00,0.50,0.00}{##1}}}
\expandafter\def\csname PY@tok@c1\endcsname{\let\PY@it=\textit\def\PY@tc##1{\textcolor[rgb]{0.25,0.50,0.50}{##1}}}
\expandafter\def\csname PY@tok@fm\endcsname{\def\PY@tc##1{\textcolor[rgb]{0.00,0.00,1.00}{##1}}}
\expandafter\def\csname PY@tok@o\endcsname{\def\PY@tc##1{\textcolor[rgb]{0.40,0.40,0.40}{##1}}}
\expandafter\def\csname PY@tok@kc\endcsname{\let\PY@bf=\textbf\def\PY@tc##1{\textcolor[rgb]{0.00,0.50,0.00}{##1}}}
\expandafter\def\csname PY@tok@c\endcsname{\let\PY@it=\textit\def\PY@tc##1{\textcolor[rgb]{0.25,0.50,0.50}{##1}}}
\expandafter\def\csname PY@tok@mf\endcsname{\def\PY@tc##1{\textcolor[rgb]{0.40,0.40,0.40}{##1}}}
\expandafter\def\csname PY@tok@err\endcsname{\def\PY@bc##1{\setlength{\fboxsep}{0pt}\fcolorbox[rgb]{1.00,0.00,0.00}{1,1,1}{\strut ##1}}}
\expandafter\def\csname PY@tok@mb\endcsname{\def\PY@tc##1{\textcolor[rgb]{0.40,0.40,0.40}{##1}}}
\expandafter\def\csname PY@tok@ss\endcsname{\def\PY@tc##1{\textcolor[rgb]{0.10,0.09,0.49}{##1}}}
\expandafter\def\csname PY@tok@sr\endcsname{\def\PY@tc##1{\textcolor[rgb]{0.73,0.40,0.53}{##1}}}
\expandafter\def\csname PY@tok@mo\endcsname{\def\PY@tc##1{\textcolor[rgb]{0.40,0.40,0.40}{##1}}}
\expandafter\def\csname PY@tok@kd\endcsname{\let\PY@bf=\textbf\def\PY@tc##1{\textcolor[rgb]{0.00,0.50,0.00}{##1}}}
\expandafter\def\csname PY@tok@mi\endcsname{\def\PY@tc##1{\textcolor[rgb]{0.40,0.40,0.40}{##1}}}
\expandafter\def\csname PY@tok@kn\endcsname{\let\PY@bf=\textbf\def\PY@tc##1{\textcolor[rgb]{0.00,0.50,0.00}{##1}}}
\expandafter\def\csname PY@tok@cpf\endcsname{\let\PY@it=\textit\def\PY@tc##1{\textcolor[rgb]{0.25,0.50,0.50}{##1}}}
\expandafter\def\csname PY@tok@kr\endcsname{\let\PY@bf=\textbf\def\PY@tc##1{\textcolor[rgb]{0.00,0.50,0.00}{##1}}}
\expandafter\def\csname PY@tok@s\endcsname{\def\PY@tc##1{\textcolor[rgb]{0.73,0.13,0.13}{##1}}}
\expandafter\def\csname PY@tok@kp\endcsname{\def\PY@tc##1{\textcolor[rgb]{0.00,0.50,0.00}{##1}}}
\expandafter\def\csname PY@tok@w\endcsname{\def\PY@tc##1{\textcolor[rgb]{0.73,0.73,0.73}{##1}}}
\expandafter\def\csname PY@tok@kt\endcsname{\def\PY@tc##1{\textcolor[rgb]{0.69,0.00,0.25}{##1}}}
\expandafter\def\csname PY@tok@sc\endcsname{\def\PY@tc##1{\textcolor[rgb]{0.73,0.13,0.13}{##1}}}
\expandafter\def\csname PY@tok@sb\endcsname{\def\PY@tc##1{\textcolor[rgb]{0.73,0.13,0.13}{##1}}}
\expandafter\def\csname PY@tok@sa\endcsname{\def\PY@tc##1{\textcolor[rgb]{0.73,0.13,0.13}{##1}}}
\expandafter\def\csname PY@tok@k\endcsname{\let\PY@bf=\textbf\def\PY@tc##1{\textcolor[rgb]{0.00,0.50,0.00}{##1}}}
\expandafter\def\csname PY@tok@se\endcsname{\let\PY@bf=\textbf\def\PY@tc##1{\textcolor[rgb]{0.73,0.40,0.13}{##1}}}
\expandafter\def\csname PY@tok@sd\endcsname{\let\PY@it=\textit\def\PY@tc##1{\textcolor[rgb]{0.73,0.13,0.13}{##1}}}

\def\PYZbs{\char`\\}
\def\PYZus{\char`\_}
\def\PYZob{\char`\{}
\def\PYZcb{\char`\}}
\def\PYZca{\char`\^}
\def\PYZam{\char`\&}
\def\PYZlt{\char`\<}
\def\PYZgt{\char`\>}
\def\PYZsh{\char`\#}
\def\PYZpc{\char`\%}
\def\PYZdl{\char`\$}
\def\PYZhy{\char`\-}
\def\PYZsq{\char`\'}
\def\PYZdq{\char`\"}
\def\PYZti{\char`\~}
% for compatibility with earlier versions
\def\PYZat{@}
\def\PYZlb{[}
\def\PYZrb{]}
\makeatother


    % Exact colors from NB
    \definecolor{incolor}{rgb}{0.0, 0.0, 0.5}
    \definecolor{outcolor}{rgb}{0.545, 0.0, 0.0}



    
    % Prevent overflowing lines due to hard-to-break entities
    \sloppy 
    % Setup hyperref package
    \hypersetup{
      breaklinks=true,  % so long urls are correctly broken across lines
      colorlinks=true,
      urlcolor=urlcolor,
      linkcolor=linkcolor,
      citecolor=citecolor,
      }
    % Slightly bigger margins than the latex defaults
    
    \geometry{verbose,tmargin=1in,bmargin=1in,lmargin=1in,rmargin=1in}
    
    

    \begin{document}
    
    
    \maketitle
    
    

    
    \section{NTDS'18 milestone 1: network collection and
properties}\label{ntds18-milestone-1-network-collection-and-properties}

\href{https://lts4.epfl.ch/simou}{Effrosyni Simou},
\href{https://lts4.epfl.ch}{EPFL LTS4}

    \subsection{Students}\label{students}

\begin{itemize}
\tightlist
\item
  Team: 31
\item
  Students: Dilara Günay, Derin Sinan Bursa, Othman Benchekroun, Sinan
  Gökçe
\item
  Dataset: IMDb Films and Crew
\end{itemize}

    \subsection{Rules}\label{rules}

\begin{itemize}
\tightlist
\item
  Milestones have to be completed by teams. No collaboration between
  teams is allowed.
\item
  Textual answers shall be short. Typically one to three sentences.
\item
  Code has to be clean.
\item
  You cannot import any other library than we imported.
\item
  When submitting, the notebook is executed and the results are stored.
  I.e., if you open the notebook again it should show numerical results
  and plots. We won't be able to execute your notebooks.
\item
  The notebook is re-executed from a blank state before submission. That
  is to be sure it is reproducible. You can click "Kernel" then "Restart
  \& Run All" in Jupyter.
\end{itemize}

    \subsection{Objective}\label{objective}

    The purpose of this milestone is to start getting acquainted to the
network that you will use for this class. In the first part of the
milestone you will import your data using
\href{http://pandas.pydata.org}{Pandas} and you will create the
adjacency matrix using \href{http://www.numpy.org}{Numpy}. This part is
project specific. In the second part you will have to compute some basic
properties of your network. \textbf{For the computation of the
properties you are only allowed to use the packages that have been
imported in the cell below.} You are not allowed to use any
graph-specific toolboxes for this milestone (such as networkx and
PyGSP). Furthermore, the aim is not to blindly compute the network
properties, but to also start to think about what kind of network you
will be working with this semester.

    \begin{Verbatim}[commandchars=\\\{\}]
{\color{incolor}In [{\color{incolor}1}]:} \PY{k+kn}{import} \PY{n+nn}{math}
        \PY{k+kn}{import} \PY{n+nn}{itertools}
        \PY{k+kn}{import} \PY{n+nn}{numpy} \PY{k+kn}{as} \PY{n+nn}{np}
        \PY{k+kn}{import} \PY{n+nn}{pandas} \PY{k+kn}{as} \PY{n+nn}{pd}
        \PY{k+kn}{import} \PY{n+nn}{matplotlib.pyplot} \PY{k+kn}{as} \PY{n+nn}{plt}
\end{Verbatim}


    \subsection{Part 1 - Import your data and manipulate
them.}\label{part-1---import-your-data-and-manipulate-them.}

    \subsubsection{A. Load your data in a Panda
dataframe.}\label{a.-load-your-data-in-a-panda-dataframe.}

    First, you should define and understand what are your nodes, what
features you have and what are your labels. Please provide below a Panda
dataframe where each row corresponds to a node with its features and
labels. For example, in the the case of the Free Music Archive (FMA)
Project, each row of the dataframe would be of the following form:

\begin{longtable}[]{@{}ccccccccc@{}}
\toprule
Track & Feature 1 & Feature 2 & . . . & Feature 518 & Label 1 & Label 2
& . . . & Label 16\tabularnewline
\midrule
\endhead
& & & & & & & &\tabularnewline
\bottomrule
\end{longtable}

It is possible that in some of the projects either the features or the
labels are not available. This is OK, in that case just make sure that
you create a dataframe where each of the rows corresponds to a node and
its associated features or labels.

    \begin{Verbatim}[commandchars=\\\{\}]
{\color{incolor}In [{\color{incolor}2}]:} \PY{n}{movies} \PY{o}{=} \PY{n}{pd}\PY{o}{.}\PY{n}{read\PYZus{}csv}\PY{p}{(}\PY{l+s+s1}{\PYZsq{}}\PY{l+s+s1}{data/tmdb\PYZus{}5000\PYZus{}credits.csv}\PY{l+s+s1}{\PYZsq{}}\PY{p}{)}
        \PY{n}{movies} \PY{o}{=} \PY{n}{movies}\PY{p}{[}\PY{n}{movies}\PY{o}{.}\PY{n}{cast} \PY{o}{!=} \PY{l+s+s1}{\PYZsq{}}\PY{l+s+s1}{[]}\PY{l+s+s1}{\PYZsq{}}\PY{p}{]}
        \PY{n}{movies}\PY{o}{.}\PY{n}{drop}\PY{p}{(}\PY{p}{[}\PY{l+m+mi}{813}\PY{p}{,} \PY{l+m+mi}{1891}\PY{p}{,} \PY{l+m+mi}{2862}\PY{p}{,} \PY{l+m+mi}{2905}\PY{p}{]}\PY{p}{,} \PY{n}{axis}\PY{o}{=}\PY{l+m+mi}{0}\PY{p}{,} \PY{n}{inplace}\PY{o}{=}\PY{n+nb+bp}{True}\PY{p}{)}
        \PY{n}{movies}\PY{o}{.}\PY{n}{head}\PY{p}{(}\PY{p}{)}
\end{Verbatim}


\begin{Verbatim}[commandchars=\\\{\}]
{\color{outcolor}Out[{\color{outcolor}2}]:}    movie\_id                                     title  \textbackslash{}
        0     19995                                    Avatar   
        1       285  Pirates of the Caribbean: At World's End   
        2    206647                                   Spectre   
        3     49026                     The Dark Knight Rises   
        4     49529                               John Carter   
        
                                                        cast  \textbackslash{}
        0  [\{"cast\_id": 242, "character": "Jake Sully", "{\ldots}   
        1  [\{"cast\_id": 4, "character": "Captain Jack Spa{\ldots}   
        2  [\{"cast\_id": 1, "character": "James Bond", "cr{\ldots}   
        3  [\{"cast\_id": 2, "character": "Bruce Wayne / Ba{\ldots}   
        4  [\{"cast\_id": 5, "character": "John Carter", "c{\ldots}   
        
                                                        crew  
        0  [\{"credit\_id": "52fe48009251416c750aca23", "de{\ldots}  
        1  [\{"credit\_id": "52fe4232c3a36847f800b579", "de{\ldots}  
        2  [\{"credit\_id": "54805967c3a36829b5002c41", "de{\ldots}  
        3  [\{"credit\_id": "52fe4781c3a36847f81398c3", "de{\ldots}  
        4  [\{"credit\_id": "52fe479ac3a36847f813eaa3", "de{\ldots}  
\end{Verbatim}
            
    \begin{Verbatim}[commandchars=\\\{\}]
{\color{incolor}In [{\color{incolor}3}]:} \PY{k}{def} \PY{n+nf}{flatten}\PY{p}{(}\PY{p}{)}\PY{p}{:}
            \PY{n}{movies}\PY{p}{[}\PY{l+s+s1}{\PYZsq{}}\PY{l+s+s1}{cast}\PY{l+s+s1}{\PYZsq{}}\PY{p}{]} \PY{o}{=} \PY{n}{movies}\PY{p}{[}\PY{l+s+s1}{\PYZsq{}}\PY{l+s+s1}{cast}\PY{l+s+s1}{\PYZsq{}}\PY{p}{]}\PY{o}{.}\PY{n}{apply}\PY{p}{(}\PY{k}{lambda} \PY{n}{row}\PY{p}{:} \PY{n+nb}{list}\PY{p}{(}\PY{n}{pd}\PY{o}{.}\PY{n}{read\PYZus{}json}\PY{p}{(}\PY{n}{row}\PY{p}{)}\PY{p}{[}\PY{l+s+s1}{\PYZsq{}}\PY{l+s+s1}{id}\PY{l+s+s1}{\PYZsq{}}\PY{p}{]}\PY{p}{)}\PY{p}{)}
            \PY{n}{movies}\PY{p}{[}\PY{l+s+s1}{\PYZsq{}}\PY{l+s+s1}{cast}\PY{l+s+s1}{\PYZsq{}}\PY{p}{]} \PY{o}{=} \PY{n}{movies}\PY{p}{[}\PY{l+s+s1}{\PYZsq{}}\PY{l+s+s1}{cast}\PY{l+s+s1}{\PYZsq{}}\PY{p}{]}\PY{o}{.}\PY{n}{apply}\PY{p}{(}\PY{k}{lambda} \PY{n}{x}\PY{p}{:} \PY{n+nb}{list}\PY{p}{(}\PY{n+nb}{set}\PY{p}{(}\PY{n}{x}\PY{p}{)}\PY{p}{)}\PY{p}{)}
            \PY{n}{movies}\PY{o}{.}\PY{n}{drop}\PY{p}{(}\PY{p}{[}\PY{l+s+s1}{\PYZsq{}}\PY{l+s+s1}{title}\PY{l+s+s1}{\PYZsq{}}\PY{p}{,} \PY{l+s+s1}{\PYZsq{}}\PY{l+s+s1}{crew}\PY{l+s+s1}{\PYZsq{}}\PY{p}{]}\PY{p}{,} \PY{n}{axis}\PY{o}{=}\PY{l+m+mi}{1}\PY{p}{,} \PY{n}{inplace}\PY{o}{=}\PY{n+nb+bp}{True}\PY{p}{)}
\end{Verbatim}


    \begin{Verbatim}[commandchars=\\\{\}]
{\color{incolor}In [{\color{incolor}4}]:} \PY{n}{flatten}\PY{p}{(}\PY{p}{)}
        \PY{n}{movies}\PY{o}{.}\PY{n}{head}\PY{p}{(}\PY{p}{)}
\end{Verbatim}


\begin{Verbatim}[commandchars=\\\{\}]
{\color{outcolor}Out[{\color{outcolor}4}]:}    movie\_id                                               cast
        0     19995  [397312, 32747, 1207263, 1180936, 1145098, 606{\ldots}
        1       285  [2603, 2440, 2441, 2449, 2450, 2452, 1430, 705{\ldots}
        2    206647  [1168129, 578512, 1599254, 1599239, 1437333, 1{\ldots}
        3     49026  [21505, 53252, 1188950, 1172491, 1343746, 7733{\ldots}
        4     49529  [1721985, 62082, 17287, 1721992, 17419, 6416, {\ldots}
\end{Verbatim}
            
    \begin{Verbatim}[commandchars=\\\{\}]
{\color{incolor}In [{\color{incolor}17}]:} \PY{n}{discount} \PY{o}{=} \PY{n}{movies}\PY{p}{[}\PY{l+s+s1}{\PYZsq{}}\PY{l+s+s1}{cast}\PY{l+s+s1}{\PYZsq{}}\PY{p}{]}\PY{o}{.}\PY{n}{apply}\PY{p}{(}\PY{n}{pd}\PY{o}{.}\PY{n}{Series}\PY{p}{)}\PY{o}{.}\PY{n}{stack}\PY{p}{(}\PY{p}{)}\PY{o}{.}\PY{n}{value\PYZus{}counts}\PY{p}{(}\PY{p}{)}
         \PY{n}{discount}\PY{o}{.}\PY{n}{head}\PY{p}{(}\PY{p}{)}
         \PY{c+c1}{\PYZsh{} 2231 id actor has playedin 67 movies}
\end{Verbatim}


\begin{Verbatim}[commandchars=\\\{\}]
{\color{outcolor}Out[{\color{outcolor}17}]:} 2231.0    67
         380.0     57
         62.0      51
         1892.0    48
         192.0     46
         dtype: int64
\end{Verbatim}
            
    \begin{Verbatim}[commandchars=\\\{\}]
{\color{incolor}In [{\color{incolor}18}]:} \PY{n}{discount} \PY{o}{=} \PY{n}{discount}\PY{p}{[}\PY{n}{discount} \PY{o}{\PYZgt{}} \PY{l+m+mi}{4}\PY{p}{]}\PY{o}{.}\PY{n}{index}\PY{o}{.}\PY{n}{astype}\PY{p}{(}\PY{n+nb}{int}\PY{p}{)}
         \PY{n}{discount}\PY{p}{[}\PY{p}{:}\PY{l+m+mi}{10}\PY{p}{]}
\end{Verbatim}


\begin{Verbatim}[commandchars=\\\{\}]
{\color{outcolor}Out[{\color{outcolor}18}]:} Int64Index([2231, 380, 62, 1892, 192, 884, 3896, 887, 85, 2963], dtype='int64')
\end{Verbatim}
            
    \begin{Verbatim}[commandchars=\\\{\}]
{\color{incolor}In [{\color{incolor}19}]:} \PY{n+nb}{len}\PY{p}{(}\PY{n}{discount}\PY{p}{)}
\end{Verbatim}


\begin{Verbatim}[commandchars=\\\{\}]
{\color{outcolor}Out[{\color{outcolor}19}]:} 3760
\end{Verbatim}
            
    \begin{Verbatim}[commandchars=\\\{\}]
{\color{incolor}In [{\color{incolor}20}]:} \PY{n}{movies}\PY{p}{[}\PY{l+s+s1}{\PYZsq{}}\PY{l+s+s1}{cast}\PY{l+s+s1}{\PYZsq{}}\PY{p}{]} \PY{o}{=} \PY{n}{movies}\PY{p}{[}\PY{l+s+s1}{\PYZsq{}}\PY{l+s+s1}{cast}\PY{l+s+s1}{\PYZsq{}}\PY{p}{]}\PY{o}{.}\PY{n}{apply}\PY{p}{(}\PY{k}{lambda} \PY{n}{x}\PY{p}{:} \PY{p}{[}\PY{n}{y} \PY{k}{for} \PY{n}{y} \PY{o+ow}{in} \PY{n}{x} \PY{k}{if} \PY{n}{y} \PY{o+ow}{in} \PY{n}{discount}\PY{p}{]}\PY{p}{)}
         \PY{n}{movies}\PY{p}{[}\PY{l+s+s1}{\PYZsq{}}\PY{l+s+s1}{edges}\PY{l+s+s1}{\PYZsq{}}\PY{p}{]} \PY{o}{=} \PY{n}{movies}\PY{p}{[}\PY{l+s+s1}{\PYZsq{}}\PY{l+s+s1}{cast}\PY{l+s+s1}{\PYZsq{}}\PY{p}{]}\PY{o}{.}\PY{n}{apply}\PY{p}{(}\PY{k}{lambda} \PY{n}{x}\PY{p}{:} \PY{n+nb}{list}\PY{p}{(}\PY{n}{itertools}\PY{o}{.}\PY{n}{combinations}\PY{p}{(}\PY{n}{x}\PY{p}{,} \PY{l+m+mi}{2}\PY{p}{)}\PY{p}{)}\PY{p}{)}
         \PY{n}{edges} \PY{o}{=} \PY{n+nb}{list}\PY{p}{(}\PY{n}{movies}\PY{p}{[}\PY{l+s+s1}{\PYZsq{}}\PY{l+s+s1}{edges}\PY{l+s+s1}{\PYZsq{}}\PY{p}{]}\PY{o}{.}\PY{n}{apply}\PY{p}{(}\PY{n}{pd}\PY{o}{.}\PY{n}{Series}\PY{p}{)}\PY{o}{.}\PY{n}{stack}\PY{p}{(}\PY{p}{)}\PY{p}{)}
         \PY{n}{edges}\PY{p}{[}\PY{l+m+mi}{0}\PY{p}{:}\PY{l+m+mi}{10}\PY{p}{]}
\end{Verbatim}


\begin{Verbatim}[commandchars=\\\{\}]
{\color{outcolor}Out[{\color{outcolor}20}]:} [(32747, 30485),
          (32747, 236696),
          (32747, 98215),
          (32747, 42286),
          (32747, 68278),
          (32747, 65731),
          (32747, 42317),
          (32747, 10964),
          (32747, 10205),
          (32747, 59231)]
\end{Verbatim}
            
    \begin{Verbatim}[commandchars=\\\{\}]
{\color{incolor}In [{\color{incolor}21}]:} \PY{n+nb}{len}\PY{p}{(}\PY{n}{edges}\PY{p}{)}
\end{Verbatim}


\begin{Verbatim}[commandchars=\\\{\}]
{\color{outcolor}Out[{\color{outcolor}21}]:} 205914
\end{Verbatim}
            
    \begin{Verbatim}[commandchars=\\\{\}]
{\color{incolor}In [{\color{incolor} }]:} \PY{n}{features} \PY{o}{=} \PY{c+c1}{\PYZsh{} Gender + matrix of movies with row= cast\PYZus{}id; column = movie\PYZus{}id}
\end{Verbatim}


    \subsubsection{B. Create the adjacency matrix of your
network.}\label{b.-create-the-adjacency-matrix-of-your-network.}

    Remember that there are edges connecting the attributed nodes that you
organized in the dataframe above. The connectivity of the network is
captured by the adjacency matrix \(W\). If \(N\) is the number of nodes,
the adjacency matrix is an \(N \times N\) matrix where the value of
\(W(i,j)\) is the weight of the edge connecting node \(i\) to node
\(j\).

There are two possible scenarios for your adjacency matrix construction,
as you already learned in the tutorial by Benjamin:

\begin{enumerate}
\def\labelenumi{\arabic{enumi})}
\item
  The edges are given to you explicitly. In this case you should simply
  load the file containing the edge information and parse it in order to
  create your adjacency matrix. See how to do that in the \href{}{graph
  from edge list} demo.
\item
  The edges are not given to you. In that case you will have to create a
  feature graph. In order to do that you will have to chose a distance
  that will quantify how similar two nodes are based on the values in
  their corresponding feature vectors. In the \href{}{graph from
  features} demo Benjamin showed you how to build feature graphs when
  using Euclidean distances between feature vectors. Be curious and
  explore other distances as well! For instance, in the case of
  high-dimensional feature vectors, you might want to consider using the
  cosine distance. Once you compute the distances between your nodes you
  will have a fully connected network. Do not forget to sparsify by
  keeping the most important edges in your network.
\end{enumerate}

Follow the appropriate steps for the construction of the adjacency
matrix of your network and provide it in the Numpy array
\texttt{adjacency} below:

    \begin{Verbatim}[commandchars=\\\{\}]
{\color{incolor}In [{\color{incolor}22}]:} \PY{n}{adj} \PY{o}{=} \PY{n}{pd}\PY{o}{.}\PY{n}{DataFrame}\PY{p}{(}\PY{n}{np}\PY{o}{.}\PY{n}{zeros}\PY{p}{(}\PY{n}{shape}\PY{o}{=}\PY{p}{(}\PY{n+nb}{len}\PY{p}{(}\PY{n}{discount}\PY{p}{)}\PY{p}{,}\PY{n+nb}{len}\PY{p}{(}\PY{n}{discount}\PY{p}{)}\PY{p}{)}\PY{p}{)}\PY{p}{,} \PY{n}{columns}\PY{o}{=}\PY{n}{discount}\PY{p}{,} \PY{n}{index}\PY{o}{=}\PY{n}{discount}\PY{p}{)}
         \PY{n}{adj}\PY{o}{.}\PY{n}{head}\PY{p}{(}\PY{p}{)}
\end{Verbatim}


\begin{Verbatim}[commandchars=\\\{\}]
{\color{outcolor}Out[{\color{outcolor}22}]:}       2231    380     62      1892    192     884     3896    887     85      \textbackslash{}
         2231     0.0     0.0     0.0     0.0     0.0     0.0     0.0     0.0     0.0   
         380      0.0     0.0     0.0     0.0     0.0     0.0     0.0     0.0     0.0   
         62       0.0     0.0     0.0     0.0     0.0     0.0     0.0     0.0     0.0   
         1892     0.0     0.0     0.0     0.0     0.0     0.0     0.0     0.0     0.0   
         192      0.0     0.0     0.0     0.0     0.0     0.0     0.0     0.0     0.0   
         
               2963     {\ldots}    58620   6069    16376   55422   17078   6165    20215   \textbackslash{}
         2231     0.0   {\ldots}       0.0     0.0     0.0     0.0     0.0     0.0     0.0   
         380      0.0   {\ldots}       0.0     0.0     0.0     0.0     0.0     0.0     0.0   
         62       0.0   {\ldots}       0.0     0.0     0.0     0.0     0.0     0.0     0.0   
         1892     0.0   {\ldots}       0.0     0.0     0.0     0.0     0.0     0.0     0.0   
         192      0.0   {\ldots}       0.0     0.0     0.0     0.0     0.0     0.0     0.0   
         
               19511   52366   968889  
         2231     0.0     0.0     0.0  
         380      0.0     0.0     0.0  
         62       0.0     0.0     0.0  
         1892     0.0     0.0     0.0  
         192      0.0     0.0     0.0  
         
         [5 rows x 3760 columns]
\end{Verbatim}
            
    \begin{Verbatim}[commandchars=\\\{\}]
{\color{incolor}In [{\color{incolor}23}]:} \PY{k}{for} \PY{n}{e1}\PY{p}{,} \PY{n}{e2} \PY{o+ow}{in} \PY{n}{edges}\PY{p}{:}
             \PY{n}{adj}\PY{o}{.}\PY{n}{at}\PY{p}{[}\PY{n}{e1}\PY{p}{,} \PY{n}{e2}\PY{p}{]} \PY{o}{+}\PY{o}{=} \PY{l+m+mi}{1}
             \PY{n}{adj}\PY{o}{.}\PY{n}{at}\PY{p}{[}\PY{n}{e2}\PY{p}{,} \PY{n}{e1}\PY{p}{]} \PY{o}{+}\PY{o}{=} \PY{l+m+mi}{1}
\end{Verbatim}


    \begin{Verbatim}[commandchars=\\\{\}]
{\color{incolor}In [{\color{incolor}24}]:} \PY{n}{adj}\PY{o}{.}\PY{n}{head}\PY{p}{(}\PY{p}{)}
\end{Verbatim}


\begin{Verbatim}[commandchars=\\\{\}]
{\color{outcolor}Out[{\color{outcolor}24}]:}       2231    380     62      1892    192     884     3896    887     85      \textbackslash{}
         2231     0.0     2.0     4.0     0.0     0.0     2.0     1.0     0.0     0.0   
         380      2.0     0.0     1.0     0.0     1.0     0.0     0.0     3.0     0.0   
         62       4.0     1.0     0.0     1.0     2.0     2.0     0.0     2.0     0.0   
         1892     0.0     0.0     1.0     0.0     1.0     0.0     1.0     0.0     0.0   
         192      0.0     1.0     2.0     1.0     0.0     0.0     4.0     1.0     1.0   
         
               2963     {\ldots}    58620   6069    16376   55422   17078   6165    20215   \textbackslash{}
         2231     2.0   {\ldots}       0.0     0.0     0.0     0.0     0.0     0.0     0.0   
         380      0.0   {\ldots}       0.0     0.0     0.0     0.0     0.0     0.0     0.0   
         62       1.0   {\ldots}       0.0     0.0     0.0     0.0     0.0     0.0     0.0   
         1892     0.0   {\ldots}       0.0     0.0     0.0     0.0     0.0     0.0     1.0   
         192      0.0   {\ldots}       0.0     0.0     0.0     0.0     0.0     0.0     0.0   
         
               19511   52366   968889  
         2231     0.0     0.0     0.0  
         380      0.0     0.0     1.0  
         62       0.0     0.0     1.0  
         1892     0.0     0.0     0.0  
         192      0.0     0.0     0.0  
         
         [5 rows x 3760 columns]
\end{Verbatim}
            
    \begin{Verbatim}[commandchars=\\\{\}]
{\color{incolor}In [{\color{incolor}25}]:} \PY{n}{adjacency} \PY{o}{=} \PY{n}{adj}\PY{o}{.}\PY{n}{values}
         \PY{n}{adj\PYZus{}max} \PY{o}{=} \PY{n}{adjacency}\PY{o}{.}\PY{n}{max}\PY{p}{(}\PY{p}{)}
         \PY{n}{adjacency} \PY{o}{=} \PY{n}{np}\PY{o}{.}\PY{n}{vectorize}\PY{p}{(}\PY{k}{lambda} \PY{n}{x}\PY{p}{:} \PY{n}{x}\PY{o}{/}\PY{n}{adj\PYZus{}max}\PY{p}{)}\PY{p}{(}\PY{n}{adjacency}\PY{p}{)}
\end{Verbatim}


    \begin{Verbatim}[commandchars=\\\{\}]
{\color{incolor}In [{\color{incolor}26}]:} \PY{n}{adjacency}
\end{Verbatim}


\begin{Verbatim}[commandchars=\\\{\}]
{\color{outcolor}Out[{\color{outcolor}26}]:} array([[0.        , 0.16666667, 0.33333333, {\ldots}, 0.        , 0.        ,
                 0.        ],
                [0.16666667, 0.        , 0.08333333, {\ldots}, 0.        , 0.        ,
                 0.08333333],
                [0.33333333, 0.08333333, 0.        , {\ldots}, 0.        , 0.        ,
                 0.08333333],
                {\ldots},
                [0.        , 0.        , 0.        , {\ldots}, 0.        , 0.        ,
                 0.        ],
                [0.        , 0.        , 0.        , {\ldots}, 0.        , 0.        ,
                 0.        ],
                [0.        , 0.08333333, 0.08333333, {\ldots}, 0.        , 0.        ,
                 0.        ]])
\end{Verbatim}
            
    \begin{Verbatim}[commandchars=\\\{\}]
{\color{incolor}In [{\color{incolor}29}]:} \PY{n}{n\PYZus{}nodes} \PY{o}{=} \PY{n+nb}{len}\PY{p}{(}\PY{n}{discount}\PY{p}{)}
         \PY{n}{n\PYZus{}nodes}
\end{Verbatim}


\begin{Verbatim}[commandchars=\\\{\}]
{\color{outcolor}Out[{\color{outcolor}29}]:} 3760
\end{Verbatim}
            
    \subsection{Part 2}\label{part-2}

    Execute the cell below to plot the (weighted) adjacency matrix of your
network.

    \begin{Verbatim}[commandchars=\\\{\}]
{\color{incolor}In [{\color{incolor}32}]:} \PY{n}{plt}\PY{o}{.}\PY{n}{spy}\PY{p}{(}\PY{n}{adjacency}\PY{p}{,} \PY{n}{markersize}\PY{o}{=}\PY{l+m+mf}{0.005}\PY{p}{)}
         \PY{n}{plt}\PY{o}{.}\PY{n}{title}\PY{p}{(}\PY{l+s+s1}{\PYZsq{}}\PY{l+s+s1}{adjacency matrix}\PY{l+s+s1}{\PYZsq{}}\PY{p}{)}
         \PY{c+c1}{\PYZsh{} because the matrix values are in descending order of movies played in: if the actor played in many movies,}
         \PY{c+c1}{\PYZsh{}his likelyhood of playing with another one increases}
\end{Verbatim}


\begin{Verbatim}[commandchars=\\\{\}]
{\color{outcolor}Out[{\color{outcolor}32}]:} Text(0.5,1.05,'adjacency matrix')
\end{Verbatim}
            
    \begin{center}
    \adjustimage{max size={0.9\linewidth}{0.9\paperheight}}{output_28_1.png}
    \end{center}
    { \hspace*{\fill} \\}
    
    \subsubsection{Question 1}\label{question-1}

What is the maximum number of links \(L_{max}\) in a network with \(N\)
nodes (where \(N\) is the number of nodes in your network)? How many
links \(L\) are there in your collected network? Comment on the sparsity
of your network.

    \begin{Verbatim}[commandchars=\\\{\}]
{\color{incolor}In [{\color{incolor}33}]:} \PY{n}{L\PYZus{}max} \PY{o}{=} \PY{n}{n\PYZus{}nodes}\PY{o}{*}\PY{p}{(}\PY{n}{n\PYZus{}nodes}\PY{o}{\PYZhy{}}\PY{l+m+mi}{1}\PY{p}{)}\PY{o}{/}\PY{l+m+mi}{2}
         \PY{k}{print}\PY{p}{(}\PY{n}{L\PYZus{}max}\PY{p}{)}
\end{Verbatim}


    \begin{Verbatim}[commandchars=\\\{\}]
7066920

    \end{Verbatim}

    \begin{Verbatim}[commandchars=\\\{\}]
{\color{incolor}In [{\color{incolor}34}]:} \PY{n}{L\PYZus{}max} \PY{o}{/} \PY{n+nb}{len}\PY{p}{(}\PY{n}{edges}\PY{p}{)}
\end{Verbatim}


\begin{Verbatim}[commandchars=\\\{\}]
{\color{outcolor}Out[{\color{outcolor}34}]:} 34
\end{Verbatim}
            
    \begin{Verbatim}[commandchars=\\\{\}]
{\color{incolor}In [{\color{incolor}35}]:} \PY{n+nb}{len}\PY{p}{(}\PY{n}{edges}\PY{p}{)}
\end{Verbatim}


\begin{Verbatim}[commandchars=\\\{\}]
{\color{outcolor}Out[{\color{outcolor}35}]:} 205914
\end{Verbatim}
            
    \begin{Verbatim}[commandchars=\\\{\}]
{\color{incolor}In [{\color{incolor}36}]:} \PY{n}{n\PYZus{}nodes} \PY{o}{*} \PY{n}{math}\PY{o}{.}\PY{n}{sqrt}\PY{p}{(}\PY{n}{n\PYZus{}nodes}\PY{p}{)}
\end{Verbatim}


\begin{Verbatim}[commandchars=\\\{\}]
{\color{outcolor}Out[{\color{outcolor}36}]:} 230558.8341400086
\end{Verbatim}
            
    \textbf{We have that the network is not fully connected, yet not sparse.
Indeed, we have that \(L=O(n*\sqrt{n})\).}

    \subsubsection{Question 2}\label{question-2}

Is your graph directed or undirected? If it is directed, convert it to
an undirected graph by symmetrizing the adjacency matrix.

    \textbf{The graph is undirected, thus we do not have to add any code.}

    \begin{Verbatim}[commandchars=\\\{\}]
{\color{incolor}In [{\color{incolor} }]:} \PY{c+c1}{\PYZsh{} \PYZhy{}\PYZhy{}}
\end{Verbatim}


    \subsubsection{Question 3}\label{question-3}

In the cell below save the features dataframe and the
\textbf{symmetrized} adjacency matrix. You can use the Pandas
\texttt{to\_csv} to save the \texttt{features} and Numpy's \texttt{save}
to save the \texttt{adjacency}. We will reuse those in the following
milestones.

    \begin{Verbatim}[commandchars=\\\{\}]
{\color{incolor}In [{\color{incolor}37}]:} \PY{n}{np}\PY{o}{.}\PY{n}{save}\PY{p}{(}\PY{l+s+s2}{\PYZdq{}}\PY{l+s+s2}{data/adj.npy}\PY{l+s+s2}{\PYZdq{}}\PY{p}{,} \PY{n}{adjacency}\PY{p}{)}
\end{Verbatim}


    \subsubsection{Question 4}\label{question-4}

Are the edges of your graph weighted?

    \textbf{Yes, the edges are weighted and the weights correspond to the
number of movies where actors played together.}

    \subsubsection{Question 5}\label{question-5}

What is the degree distibution of your network?

    \begin{Verbatim}[commandchars=\\\{\}]
{\color{incolor}In [{\color{incolor}38}]:} \PY{n}{degree} \PY{o}{=} \PY{n}{np}\PY{o}{.}\PY{n}{count\PYZus{}nonzero}\PY{p}{(}\PY{n}{adjacency}\PY{p}{,} \PY{n}{axis}\PY{o}{=}\PY{l+m+mi}{1}\PY{p}{)}
         
         \PY{k}{assert} \PY{n+nb}{len}\PY{p}{(}\PY{n}{degree}\PY{p}{)} \PY{o}{==} \PY{n}{n\PYZus{}nodes}
\end{Verbatim}


    Execute the cell below to see the histogram of the degree distribution.

    \begin{Verbatim}[commandchars=\\\{\}]
{\color{incolor}In [{\color{incolor}43}]:} \PY{n}{degree}
         \PY{c+c1}{\PYZsh{} The first row (actor with id 2231) played with 623 different actors and so on for 3760 actors}
\end{Verbatim}


\begin{Verbatim}[commandchars=\\\{\}]
{\color{outcolor}Out[{\color{outcolor}43}]:} array([623, 452, 515, {\ldots},  63,  83,  65])
\end{Verbatim}
            
    \begin{Verbatim}[commandchars=\\\{\}]
{\color{incolor}In [{\color{incolor}47}]:} \PY{c+c1}{\PYZsh{} this code is provided by assistants}
         \PY{n}{weights} \PY{o}{=} \PY{n}{np}\PY{o}{.}\PY{n}{ones\PYZus{}like}\PY{p}{(}\PY{n}{degree}\PY{p}{)} \PY{o}{/} \PY{n+nb}{float}\PY{p}{(}\PY{n}{n\PYZus{}nodes}\PY{p}{)}
         \PY{n}{plt}\PY{o}{.}\PY{n}{hist}\PY{p}{(}\PY{n}{degree}\PY{p}{,} \PY{n}{weights}\PY{o}{=}\PY{n}{weights}\PY{p}{)}\PY{p}{;}
\end{Verbatim}


    \begin{center}
    \adjustimage{max size={0.9\linewidth}{0.9\paperheight}}{output_46_0.png}
    \end{center}
    { \hspace*{\fill} \\}
    
    What is the average degree?

    \begin{Verbatim}[commandchars=\\\{\}]
{\color{incolor}In [{\color{incolor}45}]:} \PY{n}{np}\PY{o}{.}\PY{n}{sum}\PY{p}{(}\PY{n}{degree}\PY{p}{)}\PY{o}{/}\PY{n}{n\PYZus{}nodes}
         \PY{c+c1}{\PYZsh{}On average an actor plays with 98 different actors}
\end{Verbatim}


\begin{Verbatim}[commandchars=\\\{\}]
{\color{outcolor}Out[{\color{outcolor}45}]:} 98
\end{Verbatim}
            
    \subsubsection{Question 6}\label{question-6}

Comment on the degree distribution of your network.

    \textbf{The degree distribution of the graph seems large but is not
unexpected. Indeed, the number of actors in a movie is \(8\) on average,
while the number of movies an actor played in is \(10\) in average. The
order of magnitude seems to be corresponding according to the rule of
thumb.}

    \subsubsection{Question 7}\label{question-7}

Write a function that takes as input the adjacency matrix of a graph and
determines whether the graph is connected or not.

    \begin{Verbatim}[commandchars=\\\{\}]
{\color{incolor}In [{\color{incolor} }]:} \PY{k}{def} \PY{n+nf}{connected\PYZus{}graph}\PY{p}{(}\PY{n}{adjacency}\PY{p}{)}\PY{p}{:}
            
            \PY{n}{visited}\PY{p}{,} \PY{n}{queue} \PY{o}{=} \PY{n+nb}{set}\PY{p}{(}\PY{p}{)}\PY{p}{,} \PY{p}{[}\PY{p}{]}
            \PY{k}{while} \PY{n}{queue}\PY{p}{:}
                \PY{n}{actor} \PY{o}{=} \PY{n}{queue}\PY{o}{.}\PY{n}{pop}\PY{p}{(}\PY{l+m+mi}{0}\PY{p}{)}
                \PY{k}{if} \PY{n}{actor} \PY{o+ow}{not} \PY{o+ow}{in} \PY{n}{visited}\PY{p}{:}
                    \PY{n}{visited}\PY{o}{.}\PY{n}{add}\PY{p}{(}\PY{n}{actor}\PY{p}{)}
                    
                    \PY{n}{queue}\PY{o}{.}\PY{n}{extend}\PY{p}{(}\PY{n}{graph}\PY{p}{[}\PY{n}{vertex}\PY{p}{]} \PY{o}{\PYZhy{}} \PY{n}{visited}\PY{p}{)}
            \PY{k}{return} \PY{n}{visited}
        
            
            
            
            
            
            
            
            
            
            \PY{l+s+sd}{\PYZdq{}\PYZdq{}\PYZdq{}Determines whether a graph is connected.}
        \PY{l+s+sd}{    }
        \PY{l+s+sd}{    Parameters}
        \PY{l+s+sd}{    \PYZhy{}\PYZhy{}\PYZhy{}\PYZhy{}\PYZhy{}\PYZhy{}\PYZhy{}\PYZhy{}\PYZhy{}\PYZhy{}}
        \PY{l+s+sd}{    adjacency: numpy array}
        \PY{l+s+sd}{        The (weighted) adjacency matrix of a graph.}
        \PY{l+s+sd}{    }
        \PY{l+s+sd}{    Returns}
        \PY{l+s+sd}{    \PYZhy{}\PYZhy{}\PYZhy{}\PYZhy{}\PYZhy{}\PYZhy{}\PYZhy{}}
        \PY{l+s+sd}{    bool}
        \PY{l+s+sd}{        True if the graph is connected, False otherwise.}
        \PY{l+s+sd}{    \PYZdq{}\PYZdq{}\PYZdq{}}
            
            \PY{c+c1}{\PYZsh{} Your code here.}
            
            \PY{k}{return} \PY{n}{connected}
\end{Verbatim}


    Is your graph connected? Run the \texttt{connected\_graph} function to
determine your answer.

    \begin{Verbatim}[commandchars=\\\{\}]
{\color{incolor}In [{\color{incolor}66}]:} \PY{n+nb}{len}\PY{p}{(}\PY{n}{adjacency}\PY{p}{)}
\end{Verbatim}


\begin{Verbatim}[commandchars=\\\{\}]
{\color{outcolor}Out[{\color{outcolor}66}]:} 3760
\end{Verbatim}
            
    \subsubsection{Question 8}\label{question-8}

Write a function that extracts the connected components of a graph.

    \begin{Verbatim}[commandchars=\\\{\}]
{\color{incolor}In [{\color{incolor} }]:} \PY{k}{def} \PY{n+nf}{find\PYZus{}components}\PY{p}{(}\PY{n}{adjacency}\PY{p}{)}\PY{p}{:}
            \PY{l+s+sd}{\PYZdq{}\PYZdq{}\PYZdq{}Find the connected components of a graph.}
        \PY{l+s+sd}{    }
        \PY{l+s+sd}{    Parameters}
        \PY{l+s+sd}{    \PYZhy{}\PYZhy{}\PYZhy{}\PYZhy{}\PYZhy{}\PYZhy{}\PYZhy{}\PYZhy{}\PYZhy{}\PYZhy{}}
        \PY{l+s+sd}{    adjacency: numpy array}
        \PY{l+s+sd}{        The (weighted) adjacency matrix of a graph.}
        \PY{l+s+sd}{    }
        \PY{l+s+sd}{    Returns}
        \PY{l+s+sd}{    \PYZhy{}\PYZhy{}\PYZhy{}\PYZhy{}\PYZhy{}\PYZhy{}\PYZhy{}}
        \PY{l+s+sd}{    list of numpy arrays}
        \PY{l+s+sd}{        A list of adjacency matrices, one per connected component.}
        \PY{l+s+sd}{    \PYZdq{}\PYZdq{}\PYZdq{}}
            
            \PY{c+c1}{\PYZsh{} Your code here.}
            
            \PY{k}{return} \PY{n}{components}
\end{Verbatim}


    How many connected components is your network composed of? What is the
size of the largest connected component? Run the
\texttt{find\_components} function to determine your answer.

    \begin{Verbatim}[commandchars=\\\{\}]
{\color{incolor}In [{\color{incolor} }]:} \PY{c+c1}{\PYZsh{} Your code here.}
\end{Verbatim}


    \subsubsection{Question 9}\label{question-9}

Write a function that takes as input the adjacency matrix and a node
(\texttt{source}) and returns the length of the shortest path between
that node and all nodes in the graph using Dijkstra's algorithm.
\textbf{For the purposes of this assignment we are interested in the hop
distance between nodes, not in the sum of weights. }

Hint: You might want to mask the adjacency matrix in the function
\texttt{compute\_shortest\_path\_lengths} in order to make sure you
obtain a binary adjacency matrix.

    \begin{Verbatim}[commandchars=\\\{\}]
{\color{incolor}In [{\color{incolor} }]:} \PY{k}{def} \PY{n+nf}{compute\PYZus{}shortest\PYZus{}path\PYZus{}lengths}\PY{p}{(}\PY{n}{adjacency}\PY{p}{,} \PY{n}{source}\PY{p}{)}\PY{p}{:}
            \PY{l+s+sd}{\PYZdq{}\PYZdq{}\PYZdq{}Compute the shortest path length between a source node and all nodes.}
        \PY{l+s+sd}{    }
        \PY{l+s+sd}{    Parameters}
        \PY{l+s+sd}{    \PYZhy{}\PYZhy{}\PYZhy{}\PYZhy{}\PYZhy{}\PYZhy{}\PYZhy{}\PYZhy{}\PYZhy{}\PYZhy{}}
        \PY{l+s+sd}{    adjacency: numpy array}
        \PY{l+s+sd}{        The (weighted) adjacency matrix of a graph.}
        \PY{l+s+sd}{    source: int}
        \PY{l+s+sd}{        The source node. A number between 0 and n\PYZus{}nodes\PYZhy{}1.}
        \PY{l+s+sd}{    }
        \PY{l+s+sd}{    Returns}
        \PY{l+s+sd}{    \PYZhy{}\PYZhy{}\PYZhy{}\PYZhy{}\PYZhy{}\PYZhy{}\PYZhy{}}
        \PY{l+s+sd}{    list of ints}
        \PY{l+s+sd}{        The length of the shortest path from source to all nodes. Returned list should be of length n\PYZus{}nodes.}
        \PY{l+s+sd}{    \PYZdq{}\PYZdq{}\PYZdq{}}
            
            \PY{c+c1}{\PYZsh{} Your code here.}
            
            \PY{k}{return} \PY{n}{shortest\PYZus{}path\PYZus{}lengths}
\end{Verbatim}


    \subsubsection{Question 10}\label{question-10}

The diameter of the graph is the length of the longest shortest path
between any pair of nodes. Use the above developed function to compute
the diameter of the graph (or the diameter of the largest connected
component of the graph if the graph is not connected). If your graph (or
largest connected component) is very large, computing the diameter will
take very long. In that case downsample your graph so that it has 1.000
nodes. There are many ways to reduce the size of a graph. For the
purposes of this milestone you can chose to randomly select 1.000 nodes.

    \begin{Verbatim}[commandchars=\\\{\}]
{\color{incolor}In [{\color{incolor} }]:} \PY{c+c1}{\PYZsh{} Your code here.}
\end{Verbatim}


    \subsubsection{Question 11}\label{question-11}

Write a function that takes as input the adjacency matrix, a path
length, and two nodes (\texttt{source} and \texttt{target}), and returns
the number of paths of the given length between them.

    \begin{Verbatim}[commandchars=\\\{\}]
{\color{incolor}In [{\color{incolor} }]:} \PY{k}{def} \PY{n+nf}{compute\PYZus{}paths}\PY{p}{(}\PY{n}{adjacency}\PY{p}{,} \PY{n}{source}\PY{p}{,} \PY{n}{target}\PY{p}{,} \PY{n}{length}\PY{p}{)}\PY{p}{:}
            \PY{l+s+sd}{\PYZdq{}\PYZdq{}\PYZdq{}Compute the number of paths of a given length between a source and target node.}
        \PY{l+s+sd}{    }
        \PY{l+s+sd}{    Parameters}
        \PY{l+s+sd}{    \PYZhy{}\PYZhy{}\PYZhy{}\PYZhy{}\PYZhy{}\PYZhy{}\PYZhy{}\PYZhy{}\PYZhy{}\PYZhy{}}
        \PY{l+s+sd}{    adjacency: numpy array}
        \PY{l+s+sd}{        The (weighted) adjacency matrix of a graph.}
        \PY{l+s+sd}{    source: int}
        \PY{l+s+sd}{        The source node. A number between 0 and n\PYZus{}nodes\PYZhy{}1.}
        \PY{l+s+sd}{    target: int}
        \PY{l+s+sd}{        The target node. A number between 0 and n\PYZus{}nodes\PYZhy{}1.}
        \PY{l+s+sd}{    length: int}
        \PY{l+s+sd}{        The path length to be considered.}
        \PY{l+s+sd}{    }
        \PY{l+s+sd}{    Returns}
        \PY{l+s+sd}{    \PYZhy{}\PYZhy{}\PYZhy{}\PYZhy{}\PYZhy{}\PYZhy{}\PYZhy{}}
        \PY{l+s+sd}{    int}
        \PY{l+s+sd}{        The number of paths.}
        \PY{l+s+sd}{    \PYZdq{}\PYZdq{}\PYZdq{}}
            
            \PY{c+c1}{\PYZsh{} Your code here.}
            
            \PY{k}{return} \PY{n}{n\PYZus{}paths}
\end{Verbatim}


    Test your function on 5 pairs of nodes, with different lengths.

    \begin{Verbatim}[commandchars=\\\{\}]
{\color{incolor}In [{\color{incolor} }]:} \PY{k}{print}\PY{p}{(}\PY{n}{compute\PYZus{}paths}\PY{p}{(}\PY{n}{adjacency}\PY{p}{,} \PY{l+m+mi}{0}\PY{p}{,} \PY{l+m+mi}{10}\PY{p}{,} \PY{l+m+mi}{1}\PY{p}{)}\PY{p}{)}
        \PY{k}{print}\PY{p}{(}\PY{n}{compute\PYZus{}paths}\PY{p}{(}\PY{n}{adjacency}\PY{p}{,} \PY{l+m+mi}{0}\PY{p}{,} \PY{l+m+mi}{10}\PY{p}{,} \PY{l+m+mi}{2}\PY{p}{)}\PY{p}{)}
        \PY{k}{print}\PY{p}{(}\PY{n}{compute\PYZus{}paths}\PY{p}{(}\PY{n}{adjacency}\PY{p}{,} \PY{l+m+mi}{0}\PY{p}{,} \PY{l+m+mi}{10}\PY{p}{,} \PY{l+m+mi}{3}\PY{p}{)}\PY{p}{)}
        \PY{k}{print}\PY{p}{(}\PY{n}{compute\PYZus{}paths}\PY{p}{(}\PY{n}{adjacency}\PY{p}{,} \PY{l+m+mi}{23}\PY{p}{,} \PY{l+m+mi}{67}\PY{p}{,} \PY{l+m+mi}{2}\PY{p}{)}\PY{p}{)}
        \PY{k}{print}\PY{p}{(}\PY{n}{compute\PYZus{}paths}\PY{p}{(}\PY{n}{adjacency}\PY{p}{,} \PY{l+m+mi}{15}\PY{p}{,} \PY{l+m+mi}{93}\PY{p}{,} \PY{l+m+mi}{4}\PY{p}{)}\PY{p}{)}
\end{Verbatim}


    \subsubsection{Question 12}\label{question-12}

How many paths of length 3 are there in your graph? Hint: calling the
\texttt{compute\_paths} function on every pair of node is not an
efficient way to do it.

    \begin{Verbatim}[commandchars=\\\{\}]
{\color{incolor}In [{\color{incolor} }]:} \PY{c+c1}{\PYZsh{} Your code here.}
\end{Verbatim}


    \subsubsection{Question 13}\label{question-13}

Write a function that takes as input the adjacency matrix of your graph
(or of the largest connected component of your graph) and a node and
returns the clustering coefficient of that node.

    \begin{Verbatim}[commandchars=\\\{\}]
{\color{incolor}In [{\color{incolor} }]:} \PY{k}{def} \PY{n+nf}{compute\PYZus{}clustering\PYZus{}coefficient}\PY{p}{(}\PY{n}{adjacency}\PY{p}{,} \PY{n}{node}\PY{p}{)}\PY{p}{:}
            \PY{l+s+sd}{\PYZdq{}\PYZdq{}\PYZdq{}Compute the clustering coefficient of a node.}
        \PY{l+s+sd}{    }
        \PY{l+s+sd}{    Parameters}
        \PY{l+s+sd}{    \PYZhy{}\PYZhy{}\PYZhy{}\PYZhy{}\PYZhy{}\PYZhy{}\PYZhy{}\PYZhy{}\PYZhy{}\PYZhy{}}
        \PY{l+s+sd}{    adjacency: numpy array}
        \PY{l+s+sd}{        The (weighted) adjacency matrix of a graph.}
        \PY{l+s+sd}{    node: int}
        \PY{l+s+sd}{        The node whose clustering coefficient will be computed. A number between 0 and n\PYZus{}nodes\PYZhy{}1.}
        \PY{l+s+sd}{    }
        \PY{l+s+sd}{    Returns}
        \PY{l+s+sd}{    \PYZhy{}\PYZhy{}\PYZhy{}\PYZhy{}\PYZhy{}\PYZhy{}\PYZhy{}}
        \PY{l+s+sd}{    float}
        \PY{l+s+sd}{        The clustering coefficient of the node. A number between 0 and 1.}
        \PY{l+s+sd}{    \PYZdq{}\PYZdq{}\PYZdq{}}
            
            \PY{c+c1}{\PYZsh{} Your code here.}
            
            \PY{k}{return} \PY{n}{clustering\PYZus{}coefficient}
\end{Verbatim}


    \subsubsection{Question 14}\label{question-14}

What is the average clustering coefficient of your graph (or of the
largest connected component of your graph if your graph is
disconnected)? Use the function
\texttt{compute\_clustering\_coefficient} to determine your answer.

    \begin{Verbatim}[commandchars=\\\{\}]
{\color{incolor}In [{\color{incolor} }]:} \PY{c+c1}{\PYZsh{} Your code here.}
\end{Verbatim}



    % Add a bibliography block to the postdoc
    
    
    
    \end{document}
